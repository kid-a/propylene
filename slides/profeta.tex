\section{PROFETA}

\subsection{Motivation}
%------------------------------
\begin{frame}[label=1]{AMR Programming}
  %
  \begin{itemize}
    \item
      The software running on \red{A}utonomous \red{M}obile \red{R}obots
      requires the execution of many different tasks
\N  
    \item
      \red{Low level} tasks are tipically written in imperative languages 
        %(\purple{C}, \purple{assembly})
\n
    \item
      \red{Higher level} tasks, e.g. strategy control, could benefit 
      from a logic/declarative approach
\N\N  
\pause
    \item
      \navy{PROFETA} embeds declarative constructs into Python by
      using the BDI model and exploiting operators overloading
\n    
    \item
      \navy{PROFETA} provides an all-in-one environment, while combining
      the advantages of both approaches
%      into an imperative language, to provide an all-in-one environment
%    \item A combination of both paradigms seems a reasonable trade-off
  %
  \end{itemize}
% 
\N\N
\end{frame}
%------------------------------



\subsection{The BDI Model}
%------------------------------
\begin{frame}[label=2]{The Belief-Desire-Intention Model}
  %
  \begin{itemize}
    \item
      The BDI model is a philosophical theory of human reasoning 
      %[\purple{Bratman et Israel}] 
      that has been successfully applied to \red{software agents} 
      %[\purple{Wooldridge, Jennings }] 
\N
    \item 
      In particular, we have:
      %
      \begin{itemize}
      \n
        \item 
          \red{Beliefs}, that represent the \purple{knowledge} 
          about both the world and the state of the agent itself
      \n
        \item
          \red{Desires} (or \red{Goals}), that represent the 
          \purple{objectives} the agent wants to achieve
      \n
        \item
          \red{Intentions}, the sequence of \purple{actions} that allows 
          the agent to achieve a particular objective
      %
      \end{itemize} 
\N
    \item
      \navy{PROFETA} defines the corresponding Python classes
      %basic class: 
      %\purple{Belief}, \purple{Goal}, \purple{Action} and \purple{Engine}
    \end{itemize}
    %
\N\N
\end{frame}



\subsection{The Plan}
%------------------------------
\begin{frame}[label=3]{AgentSpeak(L) Plan}
  %
  \begin{itemize} 
    \item
      The \red{Plan} is a core concept of PROFETA
    \item
      It has been originally defined in AgentSpeak(L) %ref.
  %
  \end{itemize}
%%
  \begin{exampleblock}{General Syntax of an AgentSpeak(L) Plan}
    \begin{center}
      \textbf{\texttt{Trigger | Condition >> Body}}
    \end{center}
  \end{exampleblock}
%%
\N
  %
  \begin{itemize}
    \item
      \purple{Trigger} is a specific event
    \item 
      \purple{Condition} is an \emph{optional} sequence of beliefs, 
      which must hold in the agent's knowledge base
    \item
      \purple{Body} is a list that can contain both actions and events
  %
  \end{itemize}
% 
\N\N
\end{frame}
%------------------------------


%------------------------------
\begin{frame}[label=4, fragile]{PROFETA Plan}
%%
  \begin{exampleblock}{T.U.R.E. Strategy (Eurobot 2010)}
    \texttt{ 
      (  \red{+\tildett grab\_corn("c0")} | 
      ( \green{white\_corn("c0")} )  ) \\
      \tab >> [ \navy{  reach\_corn("c0"), \\ 
                \TAB pick\_corn("c0")%, \\
                     }
              ]
%        \red{ \TAB +\tildett grab\_corn("c3") } 
%        {\hskip 1.3 cm}]\\
 }
  \end{exampleblock}
%%
\N
  %
  \begin{itemize} 
    \item 
      The user must define and implement the corresponding \emph{subclass} 
      of \green{Belief}, \red{Goal} and \navy{Action}, e.g.
  \end{itemize}
  %
%%
  \begin{columns}
    \column{2.0in}
\begin{verbatim}
class white_corn(Belief):
    pass
\end{verbatim}
    \column{2.0in}
\begin{verbatim}
class reach_corn(Action):
    pass
\end{verbatim}
  \end{columns}
%%
% 
\N\N\N
\end{frame}
%------------------------------


