\section{PROFETA}

\subsection{Autonomous Mobile Robots Programming}
%------------------------------
\begin{frame}[label=1]{AMR Programming}
%{Imperative vs Declarative approach}
  \begin{itemize}
    \item
      The software running on \red{A}utonomous \red{M}obile \red{R}obots
      requires the execution of many different tasks
\N\N  
    \item
      \red{Low level} tasks are tipically written in imperative languages 
      (\purple{C}, \purple{assembly})
\N
    \item
      \red{Higher level} tasks, especially strategy control, could benefit 
      from a logic/declarative approach
\N\N  
    \item
      A combination of both paradigms seems a reasonable trade-off
    %
  \end{itemize}
% 
\N\N
\end{frame}
%------------------------------


\subsection{Avoiding the ``mixture'' of paradigms}
%------------------------------
\begin{frame}[label=2]{The ``Mixture''}
  %
  \begin{itemize}

    \item
      However, mixing different languages arises serious efficiency issues
\N
    \item
      Idea: embedding declarative constructs into an imperative language,
      to provide an all-in-one environment
\N\N
    \item 
      \navy{PROFETA} applies this idea to Python, by exploiting operators 
      overloading
\N
    \item
      \navy{PROFETA} is based on the key concepts of the 
      \red{B}elief-\red{D}esire- \red{I}ntention model.
 
    \end{itemize}
    %
\N\N
\end{frame}
%
%
%
%
%\item
%      Designing strategies with an imperative approach implies 
%      the use of a FSM
%  \begin{itemize}
%\N
%    \item
%      Lots of statements like:
%\n
%      \tab \small{``if ( \textbf{condition} ) then \textbf{do\_something()}''}
%
%\item
%      The code quickly becomes very hard to maintain.
%    \end{itemize}
%\pause
%    \N
%    \item
%    The \red{logic}/\red{declarative} approach employs a model closer to 
%    human reasoning process 
%\N    
%  \begin{itemize}
%\item
% It is impractical to implement the whole system using only declarative constructs.
%
%    \end{itemize}
%\end{itemize}
%\N\N
%\end{frame}
%
%
%
%%------------------------------
%\begin{frame}{The ``Mixture''}
%  \begin{itemize}
%\item
%We could combine the two approaches and exploit only their advantages 
%
%
%\item
%    However, mixing different languages and paradigms arises serious 
%    efficiency issues.
%\pause
%
%   \item 
%   \navy{PROFETA} adds \red{logic constructs} to Python programs, thus 
%    offering an all-in-one environment for both approaches
%    \N
%
%\item
%    \navy{PROFETA} is based on the key concepts of the \red{B}elief-\red{D}esire- \red{I}ntention model.
% 
%\item 
%The BDI model is a philosophical theory of human reasoning 
%   [\purple{Bratman et Israel}] that has been successfully applied to 
%   \red{software agents} [\purple{Wooldridge, Jennings }] 
%   \N
% 
%    \end{itemize}
%\N\N
%\end{frame}
%%------------------------------
%
%\item 
%The BDI model is a philosophical theory of human reasoning 
%   [\purple{Bratman et Israel}] that has been successfully applied to 
%   \red{software agents} [\purple{Wooldridge, Jennings }] 
%   \N
% 
%
