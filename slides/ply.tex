\section{PLY}

\subsection{Overview}
%------------------------------
\begin{frame}{Propylene and PLY}  
  %
  \begin{itemize}
    \item \navy{Propylene} is a Python classes generator:
\n
    %
    \begin{itemize}
      \item \textbf{Input:} a set of Profeta Plans (=the game strategy)
\n  
      \item \textbf{Output:} a set of subclasses of Profeta attitudes 
%\n  
%      \item \textbf{Tool:} we used \red{P}yton \red{L}ex-\red{Y}acc to
%      implement Propylene
    \end{itemize}
    %
\N
    \item \navy{Propylene} uses as parsing tool \red{PLY}, a pure-Python
    implementation of \emph{lex} and \emph{yacc}
    \item
     % \red{P}yton \red{L}ex-\red{Y}acc, 
\N\n
    \red{PLY} features include:
    %
    \begin{itemize}
%\n
      \item Support for SLR/LALR parsing
%\n  
%      \item Error checking, grammar validation
%\n  
      \item No need for a separate code-generation step
    \end{itemize}
    %
  \end{itemize}
  %
  \begin{quote}
    \begin{center}
      ``The main goal of Python Lex-Yacc is to stay faithful to the way in 
      which traditional lex/yacc tools work''
    \end{center}
  \end{quote}
  %
%
\N\N
\end{frame}
%------------------------------


\subsection{Lex}
%------------------------------
\begin{frame}{lex.py}
  %
  \begin{itemize}
    \item Info about the lex.py module
\N
    \item As an example include some Propylene regexpr 
\N
    \item 1 slide is OK
\N
    \item \ldots.....
 
  \end{itemize}
  %
%
\N\N
\end{frame}


\subsection{Yacc}
%------------------------------
\begin{frame}{yacc.py}
  %
  \begin{itemize}
    \item Info about the yacc.py
\N
    \item As an example include some Propylene productions (CFG)
\N
    \item 1 slide is OK (max 2)
\N
    \item \ldots..... 
 
  \end{itemize}
  %
%
\N\N
\end{frame}
